\documentclass[14pt, a4paper]{extarticle}

\usepackage{polyglossia} % Localization
\usepackage{fontspec}
\usepackage[T1]{fontenc}
\usepackage{array}
\usepackage{setspace}
\usepackage{amsmath}
\usepackage{multirow}
\usepackage{indentfirst}
\usepackage{ragged2e}
\usepackage{titlesec}

\begin{document}
\section{Семинар 1}\label{ux441ux435ux43cux438ux43dux430ux440-1}

\subsubsection{Симметризация}\label{ux441ux438ux43cux43cux435ux442ux440ux438ux437ux430ux446ux438ux44f}

\[A_{(ij)} = \frac{1}{2}(A_{ij}+A_{ji})\]

\subsubsection{Антисимметризация}\label{ux430ux43dux442ux438ux441ux438ux43cux43cux435ux442ux440ux438ux437ux430ux446ux438ux44f}

\[A_{[ij]} = \frac{1}{2}(A_{ij}-A_{ji})\]

Симметричная матрица: \(A^{T} = A, \ A_{ij} = A_{ji}\) Антисимметричная
матрица: \(A^{T} = -A, \ A_{ij} = -A_{ji}\)

Любой тензор второго ранга можно разложить на симм. и антисимм. части:

\[A_{[ij]} + A_{(ij)} = A_{ij}\]

\subsubsection{Задача 1}\label{ux437ux430ux434ux430ux447ux430-1}

Тензор задан своими компонентами

\[[A_{ij}] = \begin{pmatrix}
1 & 2 & 3 \\
-1 & 0 & 4 \\
3 & 5 & 2
\end{pmatrix}\]

\[\begin{gather}
A_{(ij)} = \begin{pmatrix}
1 & \frac{1}{2} & 3 \\
\frac{1}{2} & 0 & \frac{9}{2} \\
3 & \frac{9}{2} & 2
\end{pmatrix} \\
A_{[ij]} = \begin{pmatrix}
0 & \frac{3}{2} & 0 \\
-\frac{3}{2} & 0 & -\frac{1}{2} \\
0 & \frac{1}{2} & 0
\end{pmatrix}
\end{gather}\]

\subsubsection{Задача 2}\label{ux437ux430ux434ux430ux447ux430-2}

Имеется тензор второго ранга в \(\mathbb{R}^{n}\). Число независимых
компонент.

Для общ. вида: \(n^{2}\) Для симм. вида: \(\frac{n^{2}+n}{2}\) Для
антисимм. вида: \(\frac{n^{2}-n}{2}\)

\subsubsection{Задача 3}\label{ux437ux430ux434ux430ux447ux430-3}

\begin{enumerate}
\def\labelenumi{\arabic{enumi}.}
\tightlist
\item
  \(3^{3} = 27\)
\item
  \(3 + 6 + 1 = 10\)
\item
  \(1\)
\end{enumerate}

\subsubsection{Задача 4}\label{ux437ux430ux434ux430ux447ux430-4}

\(A_{ij}\) - комп симм. тензора \(B^{ij}\) - комп. антисимм. тензора
\(A_{ij}B^{ij} = 0\)

\[i,j = 1,2: \ A_{11}B^{11} + A_{12}B^{12} + A_{21}B^{21} + A_{22}B^{22} = 0\]

\subsubsection{Задача 5}\label{ux437ux430ux434ux430ux447ux430-5}

Показать, что \(A_{ij}B^{[ij]}\) зависит только от антисимметричной
части тензора \(A\).

\[A_{ij}B^{[ij]} = (A_{(ij)} + A_{[ij]})B^{[ij]} = \underbrace{A_{(ij)}B^{[ij]}}_{0} + A_{[ij]}B^{[ij]}\]

\begin{quote}
{[}!definition{]} Символ Кронекера Короткая запись единичной матрицы
\[\delta_{k}^{i} = \begin{cases}
1, i=k \\
0, i\neq k
\end{cases}\]
\end{quote}

\begin{quote}
{[}!definition{]} Символ Леви-Чивита \[\begin{gather}
\mathbb{R}^{n}, \ \epsilon^{\overbrace{ij\dots k}^{n}} = \epsilon_{\underbrace{ij\dots k}_{n}} = \begin{cases}
1, \ \text{если} \ \begin{pmatrix}
1 & 2 & \dots & n \\
i & j & \dots & k
\end{pmatrix} - \text{четная перестановка} \\
-1, \ \text{если} \ \begin{pmatrix}
1 & 2 & \dots & n \\
i & j & \dots & k
\end{pmatrix} - \text{нечетная перестановка} \\
0 \ \text{в остальных случаях}
\end{cases}
\end{gather}\] \[\begin{vmatrix}
a^{1} & b^{1} & c^{1} \\
a^{2} & b^{2} & c^{2} \\
a^{3} & b^{3} & c^{3}
\end{vmatrix} = \epsilon_{ijk}a^{i}b^{j}c^{k}\]
\end{quote}

\subsubsection{Задача 6}\label{ux437ux430ux434ux430ux447ux430-6}

\[n!\]

\subsubsection{Задача 7}\label{ux437ux430ux434ux430ux447ux430-7}

\[\epsilon_{ij}\epsilon^{ik} = a_{j}^{k}\]

\[\begin{gather}
a_{1}^{1} = \epsilon_{i 1}\epsilon^{i1} = \epsilon_{11}\epsilon^{11} + \epsilon_{21}\epsilon^{21} = 1 \\
a_{1}^{2} = \epsilon_{i 1}\epsilon^{i 2} = \epsilon_{11}\epsilon^{12} + \epsilon_{21}\epsilon^{22} = 0 \\
a_{2}^{1} = 0 \\
a_{2}^{2} = 1 \\
a_{j}^{k} = \delta_{j}^{k}
\end{gather}\]

\subsubsection{Задача 8}\label{ux437ux430ux434ux430ux447ux430-8}

\[\begin{gather}
\epsilon_{ijk}\epsilon^{ijl} = a_{k}^{l}
\end{gather}\]

\section{Семинар 2}\label{ux441ux435ux43cux438ux43dux430ux440-2}

\subsection{Тензор
объема}\label{ux442ux435ux43dux437ux43eux440-ux43eux431ux44aux435ux43cux430}

\[\begin{gather} \epsilon_{ijk}a^{i}b^{j}c^{k} \\
\text{смеш. пр.} \ \vec{a} \ \vec{b} \ \vec{c} = V \\
V = \epsilon_{ijk}a^{i}b^{j}c^{k} - \text{тензор объема} \\
V: \mathbb{R}^{3\times 3} \to \mathbb{R}
\end{gather}\]

\[\begin{gather}
c_{i} = \epsilon_{ijk}a^{j}b^{k} \\
c_{1} = \epsilon_{1jk}a^{j}b^{k} = \epsilon_{123}b^{2}c^{3} + \epsilon_{132}b^{3}c^{2} = a^{2}b^{3} - a^{3}b^{2} \\
c_{2} = \epsilon_{2ij} a^{i} b^{j} = \epsilon_{213} a^{1}b^{3} + \epsilon_{231} a^{3}b^{1} = a^{3}b^{1} - a^{1}b^{3} \\
c_{3} = \epsilon_{312}a^{1}b^{2} + \epsilon_{321}a^{2}b^{1} = a^{1}b^{2} - a^{2}b^{1}
\end{gather}\]

\[\begin{gather}
\mathbb{R}^{n} \ V = \epsilon_{ij\dots k}a^{i}_{1}a^{j}_{2}\dots a^{k}_{n} - \text{антисимм. тензор n-го ранга} \\
\langle \vec{a},\vec{b} \rangle = g_{ij}a^{i}b^{j} - \text{метр. тензор - симм. тензор 2-го ранга}
\end{gather}\]

\subsection{Поднятие и опускание
индексов}\label{ux43fux43eux434ux43dux44fux442ux438ux435-ux438-ux43eux43fux443ux441ux43aux430ux43dux438ux435-ux438ux43dux434ux435ux43aux441ux43eux432}

\[\begin{gather}
x_{i} = g_{ij}x^{j} - \text{вектор} \overset{g_{ij}}{\leftrightarrow} \text{ковектор} \\
g^{ij} = (G^{-1})^{ij} \ x^{k} = g^{ki}x_{i} = \underbrace{g^{ki}g_{ij}}_{\delta_{j}^{k}}x^{j}
\end{gather}\]

\subsubsection{Пример}\label{ux43fux440ux438ux43cux435ux440}

\[\begin{gather}
g_{ij} = \begin{bmatrix}
2 & 1 & 0 \\
1 & 2 & 0 \\
0 & 0 & 3
\end{bmatrix}, \ a_{j}^{i} = \begin{bmatrix}
1 & -1 & 0 \\
-1 & 1 & 0 \\
0 & 0 & 2
\end{bmatrix} \\
\vec{a} = (1,-5,4)^{T} \\
1) \ \tilde{a}, \vec{a} - ? \\
2) a_{ij} \\
3) g^{ij} \\
4) \text{вектор, соотв. ковек.} \ \tilde{a}
\end{gather}\]

\[\begin{gather}
a_{i} = g_{ij}a^{j} \\
a_{1} = g_{11}a^{1} + g_{12}a^{2} + g_{13}a^{3} = 2-5+0 = -3 \\
a_{2} = g_{21}a^{1} + g_{22}a^{2} + g_{23}a^{3} = \dots \\
a_{3} = g_{31}a^{1} + g_{32}a^{2} + g_{33}a^{3} = \dots \\
\begin{pmatrix}
a_{1} \\
a_{2} \\
a_{3}
\end{pmatrix} = \begin{pmatrix}
2 & 1 & 0 \\
1 & 2 & 0 \\
0 & 0 & 3
\end{pmatrix}\begin{pmatrix}
a^{1} \\
a^{2} \\
a^{3}
\end{pmatrix} = \begin{pmatrix}
2a^{1} + a^{2} \\
a^{1} + 2a^{2} \\
3a^{3}
\end{pmatrix} = \begin{pmatrix}
-3 \\
-9 \\
12
\end{pmatrix} \\
\begin{bmatrix}
a_{ij}
\end{bmatrix} = \begin{pmatrix}
2 & 1 & 0 \\
1 & 2 & 0 \\
0 & 0 & 3
\end{pmatrix}\begin{pmatrix}
1 & -1 & 0 \\
-1 & 1 & 0 \\
0 & 0 & 2
\end{pmatrix} = \begin{pmatrix}
1 & -1 & 0 \\
-1 & 1 & 0 \\
0 & 0 & 6
\end{pmatrix} \\
G^{-1} = \frac{1}{3}\begin{pmatrix}
2 & -1 & 0 \\
-1 & 2 & 0 \\
0 & 0 & 1
\end{pmatrix} \\
\begin{pmatrix}
a^{1} \\
a^{2} \\
a^{3}
\end{pmatrix} = \frac{1}{3}\begin{pmatrix}
2 & -1 & 0 \\
-1 & 2 & 0 \\
0 & 0 & 1
\end{pmatrix}\begin{pmatrix}
-3 \\
-9 \\
12
\end{pmatrix} = \begin{pmatrix}
1 \\
-5 \\
4
\end{pmatrix}
\end{gather}\]

\subsection{Инванианты
тензора}\label{ux438ux43dux432ux430ux43dux438ux430ux43dux442ux44b-ux442ux435ux43dux437ux43eux440ux430}

\begin{quote}
{[}!definition{]} Инванианты тензора Скалярные величины, которые можно
построить из компонент тензора.
\end{quote}

\[(1,0): \ |a| = \sqrt{ \langle a, a \rangle }, \ \langle a,a \rangle = g_{ij}a^{i}a^{j}\]

\[\begin{gather}
\mathrm{rank} = 2: \ g_{ij}a^{ij} = \sum a^{i}_{j} = \mathrm{Tr} \ A, \ A = [a_{j}^{i}] = \mathrm{inv} \\
\det A = \epsilon_{ijk}a_{1}^{i}a_{2}^{j}a_{3}^{k} = \mathrm{inv} \\
a^{i}_{j}a^{j}_{i} = \mathrm{inv}
\end{gather}\]

\subsubsection{Пример}\label{ux43fux440ux438ux43cux435ux440-1}

\[\begin{gather}
a_{i}^{j}, a_{j}^{i}a_{i}^{j}, \epsilon_{ijk}a_{1}^{i}a_{2}^{j}a_{3}^{k} \\
A = [a_{j}^{i}] = \begin{bmatrix}
4 & 1 & 2 \\
1 & -2 & 0 \\
2 & 0 & 3
\end{bmatrix} \\
\det A = 2\begin{vmatrix}
1 & 2 \\
-2 & 0
\end{vmatrix} + 3\begin{vmatrix}
4 & 1 \\
1 & -2
\end{vmatrix} = 8-27 = -19 \\
a^{i}_{j}a_{i}^{j} = \dots = 39
\end{gather}\]

\[\begin{gather}
T - \text{т. второго ранга} \\
TA = \lambda A \\
\text{где A - главные оси,} \ \lambda \ \text{- главные значения} \\
T^{i}_{j}a^{j} = \lambda a^{i}
\end{gather}\]

\section{Семинар 3}\label{ux441ux435ux43cux438ux43dux430ux440-3}

\[\begin{gather}
A_{j}^{i}x^{j} = \lambda x^{i} \\
\hat{A} \vec{x} = \lambda \vec{x}
\end{gather}\]

\begin{quote}
{[}!important{]} Для тензора второго ранга \(\iff\) линейный оператор
Главные оси \(\iff\) собственные векторы Главные значения \(\iff\)
собственные значения
\end{quote}

\[\begin{gather}
a_{j}^{i} = \begin{pmatrix}
1 & 1 & 1 \\
1 & 1 & 1 \\
1 & 1 & 1
\end{pmatrix} \\
\begin{vmatrix}
1-\lambda & 1 & 1 \\
1 & 1-\lambda & 1 \\
1 & 1 & 1-\lambda
\end{vmatrix} = \begin{vmatrix}
1-\lambda & 1 & 1 \\
1 & 1-\lambda & 1 \\
3-\lambda & 3-\lambda & 3-\lambda
\end{vmatrix} = \\
= (3-\lambda)\begin{vmatrix}
-\lambda & 0 & 0 \\
0 & -\lambda & 0 \\
1 & 1 & 1
\end{vmatrix} = \lambda^{2}(3-\lambda) \\
\lambda_{1} = 3, \ \lambda_{2,3} = 0 \\
\begin{pmatrix}
-2 & 1 & 1 \\
1 & -2 & 1 \\
1 & 1 & -2
\end{pmatrix} \sim 
\begin{pmatrix}
1 & 0 & -1 \\
0 & 1 & -1 \\
0 & 0 & 0
\end{pmatrix} \iff \begin{cases}
x_{1} = x_{3} \\
x_{2} = x_{3} \\
x_{3} = C
\end{cases} \\
X_{1} = \begin{pmatrix}
C \\
C \\
C
\end{pmatrix} = C\begin{pmatrix}
1 \\
1 \\
1
\end{pmatrix} \\
\begin{pmatrix}
1 & 1 & 1 \\
1 & 1 & 1 \\
1 & 1 & 1
\end{pmatrix} \sim
\begin{pmatrix}
1 & 1 & 1 \\
0 & 0 & 0 \\
0 & 0 & 0
\end{pmatrix} \implies x_{1}+x_{2}+x_{3} = 0 \\
X_{2,3} = C_{1}\begin{pmatrix}
-1 \\
1 \\
0
\end{pmatrix} + C_{2} \begin{pmatrix}
-1 \\
0 \\
1
\end{pmatrix} \\
a_{i}^{i} = \mathrm{Tr} \ [a_{j}^{i}] = \sum_{i=1}^{n} a_{i}^{i} = \sum_{i=1}^{n} \lambda_{i} = 1+1+1 = 3+0+0 = 3 \\
a_{i}^{j}a_{j}^{i} = \sum_{i=1}^{n} \lambda_{i}^{2} = 3^{2}+0+0
\end{gather}\]

\subsection{Момент импульса вращающегося твердого
тела}\label{ux43cux43eux43cux435ux43dux442-ux438ux43cux43fux443ux43bux44cux441ux430-ux432ux440ux430ux449ux430ux44eux449ux435ux433ux43eux441ux44f-ux442ux432ux435ux440ux434ux43eux433ux43e-ux442ux435ux43bux430}

\(J^{ij}\) - тензор иннерции симметрий, определяет поведение твердого
тела при вращениях.

Энергия -
\(E_{i} = \frac{1}{2}J^{ij}\omega_{i}\omega_{j} = \frac{1}{2} \omega J\omega^{T}\)
Угловой момент - \(L^{i} = J^{ij}\omega_{j}, \ L = J\omega^{T}\)

\(\omega\) - (ко)вектор угловой скорости, задает ось вращения и
\(|\omega|\) = абсолютная величина угловой скорости.

\subsubsection{Пример}\label{ux43fux440ux438ux43cux435ux440-2}

\[\begin{gather}
\omega = (1,-1,1) \\
[J^{ij}] = \begin{pmatrix}
1 & 0 & 0 \\
0 & 2 & -1 \\
0 & -1 & 3
\end{pmatrix} \\
E_{v} = \frac{1}{2}\begin{pmatrix}
1 & -1 & 1
\end{pmatrix}\begin{pmatrix}
1 & 0 & 0 \\
0 & 2 & -1 \\
0 & -1 & 3
\end{pmatrix} \begin{pmatrix}
1 \\
-1 \\
1
\end{pmatrix} = \frac{1}{2}\begin{pmatrix}
1 & -3 & 4
\end{pmatrix}\begin{pmatrix}
1 \\
-1 \\
1
\end{pmatrix} = \frac{8}{2} = 4 \\
L^{i} = \begin{pmatrix}
1 & 0 & 0 \\
0 & 2 & -1 \\
0 & -1 & 3
\end{pmatrix} \begin{pmatrix}
1 \\
-1 \\
1
\end{pmatrix} \begin{pmatrix}
1 \\
-3 \\
4
\end{pmatrix}
\end{gather}\]

\subsection{Тензорное произведение
векторов}\label{ux442ux435ux43dux437ux43eux440ux43dux43eux435-ux43fux440ux43eux438ux437ux432ux435ux434ux435ux43dux438ux435-ux432ux435ux43aux442ux43eux440ux43eux432}

\[\begin{gather}
\vec{a} = a^{i}\vec{e}_{j} \\
\vec{b} = b^{j}\vec{e}_{i} \\
\vec{a}\otimes \vec{b} = (a^{j}\vec{e}_{i},b^{i}\vec{e}_{j}) = \underbrace{a^{i}b^{j}}_{f^{ij}}\underbrace{(\vec{e}_{i}, \vec{e}_{j})}_{\text{базис}}
\end{gather}\]

\subsubsection{Пример}\label{ux43fux440ux438ux43cux435ux440-3}

\[\vec{a}\otimes \vec{b}, \vec{b}\otimes \vec{a}, \vec{a} \wedge \vec{b}, \ \vec{a} = (1,2,3), \ \vec{b} = (-1,1,-1)
\]

\[\begin{gather}
\vec{a}\otimes \vec{b} = \begin{pmatrix}
1 \\
2 \\
3
\end{pmatrix}\begin{pmatrix}
-1 & 1 & -1
\end{pmatrix} = \begin{pmatrix}
-1 & 1 & -1 \\
-2 & 2 & -2 \\
-3 & 3 & -3
\end{pmatrix} \\
\vec{b} \otimes \vec{a} = \begin{pmatrix}
-1 \\
1 \\
-1
\end{pmatrix}\begin{pmatrix}
1 & 2 & 3
\end{pmatrix} = \begin{pmatrix}
-1 & -2 & -3 \\
1 & 2 & 3 \\
-1 & -2 & -3
\end{pmatrix} \\
\vec{a} \wedge \vec{b} = \vec{a} \otimes \vec{b} - \vec{b} \otimes \vec{a} = \begin{pmatrix}
0 & 3 & 2 \\
-3 & 0 & -5 \\
-2 & 5 & 0
\end{pmatrix} - \text{антисимм.}
\end{gather}\]

\subsubsection{\texorpdfstring{Всегда ли тензор общего вида типа
\((2,0)\) в \(\mathbb{R}^{n}\) можно представить в виде тензорного
произведения двух
векторов?}{Всегда ли тензор общего вида типа (2,0) в \textbackslash mathbb\{R\}\^{}\{n\} можно представить в виде тензорного произведения двух векторов?}}\label{ux432ux441ux435ux433ux434ux430-ux43bux438-ux442ux435ux43dux437ux43eux440-ux43eux431ux449ux435ux433ux43e-ux432ux438ux434ux430-ux442ux438ux43fux430-20-ux432-mathbbrn-ux43cux43eux436ux43dux43e-ux43fux440ux435ux434ux441ux442ux430ux432ux438ux442ux44c-ux432-ux432ux438ux434ux435-ux442ux435ux43dux437ux43eux440ux43dux43eux433ux43e-ux43fux440ux43eux438ux437ux432ux435ux434ux435ux43dux438ux44f-ux434ux432ux443ux445-ux432ux435ux43aux442ux43eux440ux43eux432}

Ответ - невозможно. (\(n^{2} > 2n, (n>2)\)). Даже в \(\mathbb{R}^{2}\) -
нельзя.

\subsubsection{Про
объем}\label{ux43fux440ux43e-ux43eux431ux44aux435ux43c}

\[\begin{gather}
\mathbb{R}^{2}: \\
\tilde{u} = u_{i}\tilde{e}^{i} = u_{i} dx^{i} \\
\tilde{v} = v_{j}dx^{j} \\
(u_{i}dx^{i}) \wedge (v_{j}dx^{j}) = (u_{1}dx^{1} + u_{2}dx^{2}) \wedge (v_{1}dx^{1} + v_{2}dx^{2}) = \\
= u_{1}v_{1}(dx^{1}\wedge dx^{2}) + \dots = (u_{1}v_{2} - u_{2}v_{1})dx^{1}\wedge dx^{2} = \\
= \begin{vmatrix}
u_{1} & v_{1} \\
u_{2} & v_{2}
\end{vmatrix}dx^{1}\wedge dx^{2}
\end{gather}\]

\[\begin{gather}
\mathbb{R}^{3}: \\
\tilde{u}\wedge \tilde{v}\wedge \tilde{w} = \dots = \\
= u_{i}v_{j}w_{k}\underbrace{dx^{i} \wedge dx^{j} \wedge dx^{k}}_{\epsilon^{ijk}dx^{1} \wedge dx^{2} \wedge dx^{3}} = \\
\underbrace{\epsilon^{ijk}u_{i}v_{j}w_{k}}_{V \ \text{параллелерипеда на} \ \tilde{u}, \tilde{v}, \tilde{w}}\cdot dx^{1} \wedge dx^{2} \wedge dx^{3}
\end{gather}\]

\subsubsection{Основываясь на предыдущей задаче, получить формулу замены
координат в 3-х мерном
интеграле.}\label{ux43eux441ux43dux43eux432ux44bux432ux430ux44fux441ux44c-ux43dux430-ux43fux440ux435ux434ux44bux434ux443ux449ux435ux439-ux437ux430ux434ux430ux447ux435-ux43fux43eux43bux443ux447ux438ux442ux44c-ux444ux43eux440ux43cux443ux43bux443-ux437ux430ux43cux435ux43dux44b-ux43aux43eux43eux440ux434ux438ux43dux430ux442-ux432-3-ux445-ux43cux435ux440ux43dux43eux43c-ux438ux43dux442ux435ux433ux440ux430ux43bux435.}

\[\begin{gather}
y^{i} = y^{i}(x) - \text{новые координаты} \\
dy^{1} = \frac{\partial y^{1}}{\partial x^{j}}dx^{j} \\
dy^{1} \wedge dy^{2} \wedge dy^{3} = \left( \frac{\partial y^{1}}{\partial x^{j}}dx^{j} \right) \wedge\left( \frac{\partial y^{2}}{\partial x^{k}}dx^{k} \right)\wedge\left( \frac{\partial y^{2}}{\partial x^{e}}dx^{e} \right) = \\
= \frac{\partial y^{1}}{\partial x^{j}} \cdot \frac{\partial y^{2}}{\partial x^{k}} \cdot\frac{\partial y^{3}}{\partial x^{e}} \cdot dx^{j} \wedge dx^{k} \wedge dx^{e} = \det \left( \frac{\partial y^{i}}{\partial x^{j}} \right)dx^{j} \wedge dx^{k} \wedge dx^{e}
\end{gather}\]

\section{Семинар 4}\label{ux441ux435ux43cux438ux43dux430ux440-4}

\subsection{Звезда
Ходжа}\label{ux437ux432ux435ux437ux434ux430-ux445ux43eux434ux436ux430}

\begin{quote}
{[}!definition{]} Звезда Ходжа \[\begin{gather}
(*T)_{\underbrace{i\dots l}_{n-q}} = \frac{1}{q!}\epsilon_{\underbrace{i\dots lj\dots k}_{n}}T^{\underbrace{j\dots k}_{q}}, \ (0,n-1) \overset{*}{\leftrightarrow} (q,0), \\
\text{где} \ *T - \text{тензор, дуальный} \ T
\end{gather}\]
\end{quote}

\[\begin{gather}
C_{n}^{q} = C_{n}^{n-q} = \frac{n!}{q!(n-q)!} \\
**T = (-1)^{q(n-q)}T \\
\end{gather}\]

\[\begin{gather}
\mathbb{R}^{2}: *(dx^{i}\wedge dx^{j}) = \epsilon^{ij} \\
\mathbb{R}^{3}: *(dx^{i}\wedge dx^{j}) = \epsilon^{ijk}\vec{e}_{k} \\
*\vec{e}_{i} = \frac{1}{2}\epsilon_{ijk}dx^{j}\wedge dx^{k}
\end{gather}\]

\[\begin{gather}
*(\tilde{u}\wedge\tilde{v}) = \vec{w} = \vec{u} \times \vec{v} \\
*(\tilde{u}\wedge\tilde{v}\wedge\tilde{w}) = \langle \tilde{u},\tilde{v},\tilde{w} \rangle = V
\end{gather}\]

\subsubsection{Примеры}\label{ux43fux440ux438ux43cux435ux440ux44b}

\begin{enumerate}
\def\labelenumi{\arabic{enumi}.}
\tightlist
\item
  Найти \(*((2dx^{1} + 3dx^{2})\wedge(dx^{1} - dx^{2}))\) в
  \(\mathbb{R}^{2}\) и \(\mathbb{R}^{3}\)
\end{enumerate}

\[\begin{gather}
*(2dx^{1}\wedge dx^{1} - 2dx^{1}\wedge dx^{2} + 3dx^{2}\wedge dx^{1} - 3dx^{2}\wedge dx^{2}) = *(-5dx^{1} \wedge dx^{2}) \\
\mathbb{R}^{2}: \ -5 \\
\mathbb{R}^{3}: \ -5e^{123}\vec{e_{3}} = -5\vec{e_{3}} \\
\end{gather}\]

\begin{enumerate}
\def\labelenumi{\arabic{enumi}.}
\setcounter{enumi}{1}
\tightlist
\item
  Найти
  \(*((dx^{1}-3dx^{2})\wedge(dx^{1}-dx^{2})\wedge(dx^{1}-4dx^{3}))\) в
  \(\mathbb{R}^{3}, \ \mathbb{R}^{4}\)
\end{enumerate}

\[\begin{gather}
*((dx^{1}-3dx^{2})\wedge(dx^{1}-dx^{2})\wedge(dx^{1}-4dx^{3})), \ \mathbb{R}^{3}, \ \mathbb{R}^{4} \\
\mathbb{R}^{3}: \ -8 \\
\mathbb{R}^{4}: \ -9\epsilon^{1234}\vec{e}_{4} = -8\vec{e}_{4}
\end{gather}\]

\begin{enumerate}
\def\labelenumi{\arabic{enumi}.}
\setcounter{enumi}{2}
\tightlist
\item
  Найти
  \(*((-2dx^{1}+2dx^{2})\wedge(dx^{1}\wedge dx^{2} + 4dx^{1}\wedge dx^{3}))\)
  в \(\mathbb{R}^{3}\) и \(\mathbb{R}^{4}\)
\end{enumerate}

\[\begin{gather}
*((-2dx^{1}+2dx^{2})\wedge(dx^{1}\wedge dx^{2} + 4dx^{1}\wedge dx^{3})) = *(8dx^{2}\wedge dx^{1}\wedge dx^{3}) = \\
*(-8dx^{1}\wedge dx^{2}\wedge dx^{3}) \\
\mathbb{R}^{3}: \ -8\epsilon^{123} = -8 \\
\mathbb{R}^{4}: \ -8\epsilon^{1234}\vec{e}_{4} = -8\vec{e}_{4} \\
\end{gather}\]

\begin{enumerate}
\def\labelenumi{\arabic{enumi}.}
\setcounter{enumi}{3}
\tightlist
\item
  Показать, что произведение 1 и 2-форм в \(\mathbb{R}^{3}\)
  \(\tilde{\omega}_{a} = a_{i}dx^{i}, \ \tilde{\omega}_{b} = b_{ij}dx^{i}\wedge dx^{j}\)
  выражается через скалярное произведение векторов \(a, b\) (дуальны в
  2-форме).
\end{enumerate}

\[\begin{gather}
\tilde{w}_{a} \wedge \tilde{w}_{a} = a_{k}dx^{k} \wedge b_{ij}dx^{i} \wedge dx^{j} = a_{k}b_{ij}dx^{k}\wedge dx^{i} \wedge dx^{j} = \\
= e^{kij}a_{k}b_{ij} dx^{1}\wedge dx^{2} \wedge dx^{3} = a_{k}b^{k}dx^{1}\wedge dx^{2} \wedge dx^{3} \\
(b^{k} = \epsilon^{kij}b_{ij})
\end{gather}\]

\subsection{Внешнее дифференцииорвание
(производная)}\label{ux432ux43dux435ux448ux43dux435ux435-ux434ux438ux444ux444ux435ux440ux435ux43dux446ux438ux438ux43eux440ux432ux430ux43dux438ux435-ux43fux440ux43eux438ux437ux432ux43eux434ux43dux430ux44f}

\begin{quote}
{[}!definition{]} Внешняя производная \(\tilde{d}\) - действие,
аналогичное дифференциированию, переводящее антисимметричный тензор в
антисимметричный тензор. Пусть \(\tilde{\alpha}\) - \(p\)-форма,
\(\tilde{\beta},\tilde{\gamma}\) - \(q\)-формы. 1.
\(\tilde{d}(\tilde{\beta} + \tilde{\gamma}) = \tilde{d}\tilde{\beta} + \tilde{d}\tilde{\gamma}\)
- линейность 2.
\(\tilde{d}(\tilde{\alpha} \wedge \tilde{\beta}) = d\tilde{\alpha} \wedge \tilde{\beta} + (-1)^{p} \tilde{\alpha} \wedge \tilde{d}\tilde{\beta}\)
3. \(\tilde{d}\tilde{d}\tilde{\alpha} = 0\)

\(f\) - 0-форма
\(\tilde{d}f = df = \frac{\partial f}{\partial x^{1}}dx^{1} + \frac{\partial f}{\partial x^{2}}dx^{2}\)
- 1-форма
\(\tilde{d}^{2}f = \frac{\partial^{2} f}{\partial x^{1}\partial x^{1}}(dx^{1})^{2} + 2\frac{\partial^{2} f}{\partial x^{1}\partial x^{2}}dx^{1}dx^{2} + \frac{\partial^{2} f}{\partial x^{2}\partial x^{2}}(dx^{2})^{2} = 0\)
- 2-форма
\end{quote}

\[\begin{gather}
f \in \mathbb{R}^{n}: \mathrm{grad} \ f - \tilde{d}f \mathbb{R}^{n} \\
f \in \mathbb{R}^{3}: \mathrm{rot} \ \tilde{a} = *\tilde{d}\tilde{a} \\
\underset{(0,1)}{\tilde{a}} \to \underset{(0,2)}{\tilde{d}\tilde{a}} \overset{*}{\to} \underset{(1,0)}{*\tilde{d}\tilde{a}} \\
f \in \mathbb{R}^{n} : \mathrm{div} \ \vec{a} = *\tilde{d}*\vec{a} \\
\left( \mathrm{div} \ \vec{a} = \sum_{i} \frac{\partial a^{i}}{\partial x^{i}} \right)
\end{gather}\]

\[\tilde{d}(f\tilde{d}\tilde{g}) = \tilde{d}f \wedge \tilde{d}\tilde{g}\]

\subsubsection{Примеры}\label{ux43fux440ux438ux43cux435ux440ux44b-1}

\[\begin{gather}
*\tilde{d}((x^{1})^{2}dx^{1} + x^{1}x^{2}dx^{2} + x^{1}x^{3}dx^{3}) \ \text{в} \ \mathbb{R}^{3} \\
*(2x^{1}dx^{1} \wedge dx^{1} + (x^{1}dx^{2} + x^{2}dx^{1})\wedge dx^{2} + (x^{3}dx^{1} + x^{1}dx^{3})\wedge dx^{3}) = \\
= *(x^{2}dx^{1}\wedge dx^{2} + x^{3}dx^{1} \wedge dx^{3}) = x^{2} *(dx^{1}\wedge dx^{2}) + x^{3}*(dx^{1} \wedge dx^{3}) = \\
= x^{2}\vec{e}_{3} - x^{3}\vec{e}_{2}
\end{gather}\]

\[\begin{gather}
\mathrm{div} \vec{a} = *\tilde{d}*(\vec{e}_{1}+x^{1}x^{2}\vec{e}_{2} + x^{1}x^{3}\vec{e}_{3}) = \\
= *\tilde{d}(dx^{2} \wedge dx^{3} - x^{1}x^{2}dx^{1}\wedge dx^{3} + x^{1}x^{3}dx^{1}\wedge dx^{2}) = \\
= *(0 - (x^{1}dx^{2} + x^{2}dx^{1}) \wedge dx^{1}\wedge dx^{3} + (x^{1}dx^{3} + x^{3}dx^{1})dx^{1}\wedge dx^{2}) = \\
= *(-x^{}dx^{2}\wedge dx^{12}\wedge dx^{3} + x^{1}dx^{3}\wedge dx^{1}\wedge dx^{2}) = 2x^{1}
\end{gather}\]

\section{Семинар 5}\label{ux441ux435ux43cux438ux43dux430ux440-5}

Доказать, что

\[\begin{gather}
\mathrm{div} \ \mathrm{rot} \ \vec{a} = 0 \\
\mathrm{rot} \ \mathrm{grad} \ \vec{a} = 0
\end{gather}\]

\[\begin{gather}
\mathrm{grad} \ f = \tilde{d}f = df \\
\mathrm{rot} \ \vec{a} = *\tilde{d}\tilde{a} \\
\mathrm{div} \ \vec{a} = *\tilde{d}*\vec{a} \\
\mathrm{div} \ \mathrm{rot} \ \vec{a} = *\tilde{d}*(*\tilde{d}\tilde{a}) = *\tilde{d}\tilde{d}\tilde{a} = *0 = 0 \\
\mathrm{rot} \ \mathrm{grad} \ \vec{a} = *\tilde{d}(\tilde{d}f) = *0 = 0
\end{gather}\]

\begin{quote}
{[}!definition{]} Интегрирование дифф. форм \[\begin{gather}
\int _{U} \tilde{\alpha} = \int \alpha(x)dx^{1}\wedge \dots \wedge dx^{p} = \int \alpha(x)dx^{1}\dots dx^{p} \\
\tilde{\alpha} = \alpha(x)dx^{1}\wedge \dots \wedge dx^{p}, \\
\end{gather}\] где \(\tilde{\alpha}\) - \(p\)-форма в \(p\)-мерном
пространстве, \(u\) - \(p\)-мерная область
\end{quote}

Получить формулу замены координат \(y = y(x)\) в двойном интеграле:

\[\begin{gather}
y^{i} = y^{i}(x) \\
dy^{1} \wedge dy^{2} = \det \left( \frac{\partial y^{i}}{\partial x^{i}} \right) dx^{1} \wedge dx^{2} \\
dy^{1} \wedge dy^{2} = \left( \frac{\partial y^{1}}{\partial x^{i}}dx^{i} \right) \wedge \left( \frac{\partial y^{1}}{\partial x^{j}}dx^{j} \right) = \frac{\partial y^{1}}{\partial x^{i}}\frac{\partial y^{1}}{\partial x^{i}} dx^{i} \wedge dx^{j} = \\
= \epsilon^{12}\frac{\partial y^{1}}{\partial x^{1}}\frac{\partial y^{1}}{\partial x^{2}} dx^{1} \wedge dx^{2} = \det \left( \frac{\partial y^{i}}{\partial x^{i}} \right) dx^{1} \wedge dx^{2} \\
\end{gather}\]

\begin{quote}
{[}!important{]} Теорема Стокса \[\begin{gather}
\int _{U} \tilde{d}\tilde{\alpha} = \int _{\partial U} \tilde{\alpha},
\end{gather}\] где \(\tilde{\alpha}\) - \(p\)-форма,
\(\tilde{d}\tilde{\alpha}\) - \(p+1\)-форма, \(U\) - \(p+1\)-мерная
область, \(\partial U\) - граница области \(U\), \(p\)-мерная
\end{quote}

\subsubsection{Пример}\label{ux43fux440ux438ux43cux435ux440-4}

\(\mathbb{R}^{1}\), \(0\)-форма \(f(x)\), \(n=1, \ p=0\)
\[\begin{gather}
\int df = \int_{\partial U} f \\
\int_{a}^{b} df = f(b) - f(a)
\end{gather}\]

\subsubsection{3}\label{section}

Какую формулу надо подставить в теореме Стокса, чтобы получить аналог
формулы Остроградското-Гаусса в \(\mathbb{R}^{4}\)? Записать для
\(\mathbb{R}^{4}\).

\[\begin{gather}
\iiint_{U} \mathrm{div} \ \vec{a} \, dV = \iint_{\partial U} (\vec{a}, d\vec{\Sigma}) \\
\int_{U} \tilde{d} \tilde{\alpha}  = \int _{U} \tilde{d}*\tilde{\vec{a}} = \int_{U} \mathrm{div} \ \vec{a} \, dV = \int_{\partial U} *\vec{a} \\
\begin{bmatrix}
*\tilde{d}*\vec{a} = \mathrm{div} \ \vec{a} \\
\tilde{d}*\vec{a} = \mathrm{div} \ \vec{a} \, dV \\
\end{bmatrix}
\end{gather}\]

В \(\mathbb{R}^{4}\):

\[\begin{gather}
\iiiint_{U} \mathrm{div} \ \vec{a} \, dx^{1}\wedge dx^{2} \wedge dx^{3} \wedge dx^{4} = \iiint_{\partial U} a^{i}\epsilon_{ijkl} \, dx^{j} \wedge dx^{k} \wedge dx^{l}
\end{gather}\]

\subsubsection{4}\label{section-1}

На основе решения предыдущей задачи,

\[\begin{gather}
\int_{U} \mathrm{div} \ \vec{a} \, dS = \int_{\partial  U} *\vec{a}
\end{gather}\]

получить формулу Грина, перейдя от вектора \(\vec{a}\) к дуальному ему
ковектору \(\tilde{\alpha} = *\vec{a}\).

\[\begin{gather}
\iint_{U} \left( \frac{\partial a^{1}}{\partial x^{1}} + \frac{\partial a^{2}}{\partial x^{2}} \right) dx^{1} \wedge dx^{2} \\
\begin{bmatrix}
\tilde{\alpha} = *\vec{a} = a^{i}*\vec{e}_{i} = a^{i}\epsilon_{ij} dx^{j} = \alpha_{j}dx^{j} \\
\alpha_{1} = a^{2}\epsilon_{21} = -a^{2} \\
\alpha_{2} = a^{1}\epsilon_{12} = a^{1} \\
\end{bmatrix} \\
\iint_{U} \left( \frac{\partial a^{1}}{\partial x^{1}} + \frac{\partial a^{2}}{\partial x^{2}} \right) dx^{1} \wedge dx^{2} = \iint_{U} \left( \frac{\partial \alpha^{2}}{\partial x^{1}} - \frac{\partial \alpha^{1}}{\partial x^{2}} \right) dx^{1} \wedge dx^{2} = \\
= \oint \tilde{\alpha} = \oint \left(\alpha_{1}dx^{1} + \alpha_{2}dx^{2} \right)
\end{gather}\]

\begin{quote}
{[}!definition{]} Псевдоевклидовы пространства \[\begin{gather}
g_{ik} = \begin{pmatrix}
1 & 0 \\
0 & -1
\end{pmatrix} - \text{пр-во Минковского} \\
g_{\mu v} = \begin{pmatrix}
1 & 0 & 0 & 0 \\
0 & -1 & 0 & 0 \\
0 & 0 & -1 & 0 \\
0 & 0 & 0 & -1
\end{pmatrix} \\
\end{gather}\]

\[\begin{gather}
x^{0} = ct \\
x^{1}
x^{2}
x^{3}
\end{gather}\]
\end{quote}

\subsubsection{5}\label{section-2}

\[\begin{gather}
A_{i} = g_{ij}A^{j} = g_{ji}A^{i} \\
F_{ij} = g_{ik}g_{jl}F^{kl} = g_{ii}g_{jj} F^{ij} \\
A_{1} = g_{11}A^{1} = A^{1} \\
A_{2} = g_{22}A^{2} = -2 \\
F_{11} = g_{11}g_{11}F^{11} = F^{11} \\
F_{12} = g_{11}g_{22}F^{12} = -F^{12} \\
F_{21} = g_{22}g_{11}F^{21} = -F^{21} \\
F_{22} = g_{22}g_{22}F^{22} = F^{22} \\
\end{gather}\]

В пространстве Минковского:

\[\begin{gather}
F_{12} = g_{11}g_{22} = -F^{12} \\
F_{23} = g_{22}g_{33}F^{23} = F^{23}
\end{gather}\]

\subsection{К/р}\label{ux43aux440}

\begin{enumerate}
\def\labelenumi{\arabic{enumi}.}
\tightlist
\item
  Симметричные и антисимметричные тензоры
\item
  Главные значения, главные оси, инварианты тензора
\item
  (и 4) Внешнее произведение тензоров, диффеернциирование тензоров,
  звезда Ходжа
\item
  Поднимание и опускание индексов
\end{enumerate}

\section{Семинар 6}\label{ux441ux435ux43cux438ux43dux430ux440-6}

\begin{quote}
{[}!definition{]} Поливекторы \(p\)-вектор - полностью антисимметричный
тензор типа \((p,0)\).
\end{quote}

\subsubsection{Примеры}\label{ux43fux440ux438ux43cux435ux440ux44b-2}

\((1,0)\) - вектор \((2, 0)\) - бивектор \((3, 0)\) - тривектор

\begin{quote}
{[}!definition{]} Свертка поливекторов Вектор:
\(\langle \vec{x}, \vec{y} \rangle = g_{ij}x^{i}y^{j} = \mathrm{inv}\)
Бивектор:
\(\langle \vec{x}, \vec{y} \rangle = g_{ij}g_{kj}x^{ik}y^{jl} = \mathrm{inv}\)
(свертка)
\end{quote}

\begin{quote}
{[}!definition{]} Норма вектора
\[|x| = \sqrt{ \langle x,x \rangle } = V \ (\text{н-мерный})\]
\end{quote}

Если \(v = v_{1} \wedge v_{2} \wedge \dots \wedge v_{n}\),
\(v = v_{1} \wedge v_{2} \wedge \dots \wedge v_{n}\), то

\[\begin{gather}
\langle u,v \rangle = \begin{vmatrix}
\langle u_{1},v_{1} \rangle & \langle u_{1},v_{2} \rangle & \dots & \langle u_{1}v_{n} \rangle \\
\langle u_{2},v_{1} \rangle & \langle u_{2},v_{2} \rangle & \dots & \langle u_{2}v_{n} \rangle \\
\vdots & \vdots & \ddots & \vdots \\
\langle u_{n},v_{1} \rangle & \langle u_{n},v_{2} \rangle & \dots & \langle u_{n}v_{n} \rangle \\
\end{vmatrix}
\end{gather}\]

\[\begin{gather}
|u|^{2} = \langle u,u \rangle \begin{vmatrix}
\langle u_{1},u_{1} \rangle & \dots & \langle u_{1},u_{n} \rangle \\
\dots & \dots & \dots \\
\langle u_{n},u_{1} \rangle & \dots & \langle u_{n},u_{n} \rangle \\
\end{vmatrix} - \\
- \text{определитель матрицы Грама системы} \ u_{1}, u_{2}, \dots, u_{n} \\
\end{gather}\]

\subsubsection{Задача 1}\label{ux437ux430ux434ux430ux447ux430-1-1}

Найти форму бивектора \(u = u_{1} \wedge u_{2}\).

\[\begin{gather}
u_{1} = (1,2,3)^{T} \ u_{2} = (4,0,1)^{T} \\
|u|^{2} = \begin{vmatrix}
14 & 7 \\
7 & 17
\end{vmatrix} = 189, \ |u| = \sqrt{ 189 } = 3\sqrt{ 21 }
\end{gather}\]

\subsubsection{Задача 2}\label{ux437ux430ux434ux430ux447ux430-2-1}

Найти площадь параллелограма построенного на векторах
\(u_{1} = (1,4,3,0)^{T}, \ u_{2} = (1,2,0,1)^{T}\).

\[\begin{gather}
|u|^{2} = \begin{vmatrix}
26 & 9 \\
9 & 6
\end{vmatrix} = 75, \ |u| = 5\sqrt{ 3 }
\end{gather}\]

\subsubsection{Задача 3}\label{ux437ux430ux434ux430ux447ux430-3-1}

Найти объем параллелепипеда, построенного на векторах
\(u_{1} = (1,2,3,0)^{T}, \ u_{2} = (1,1,-1,0)^{T}, \ u_{3} = (1,1,0,2)^{T}\).

\[\begin{gather}
|u|^{2} = \begin{vmatrix}
14 & 0 & 3 \\
0 & 3 & 2 \\
3 & 2 & 6
\end{vmatrix} = 14\cdot 14 + 3\cdot (-9) = 169 \\
|u| = \sqrt{ 169 } = 13 \\
\end{gather}\]

\subsubsection{Задача 4}\label{ux437ux430ux434ux430ux447ux430-4-1}

Найти кривизны кривой
\(\alpha(t) = \left( t, \frac{t^{2}}{2}-t, \frac{t^{3}}{3}, \frac{t^{4}}{4} \right)^{T}\)
при \(t = 0\).

\[\begin{gather}
k_{1} = \frac{|\dot{\alpha} \wedge \ddot{\alpha}|}{|\dot{\alpha}|^{3}} \\
k_{2} = \frac{|\dot{\alpha} \wedge \ddot{\alpha} \wedge \dddot{\alpha}|}{|\dot{\alpha} \wedge \ddot{\alpha}|^{2}} \\
k_{3} = \frac{*\left( \dot{\alpha} \wedge \ddot{\alpha} \wedge \dddot{\alpha} \wedge \alpha^{(4)} \right)|\dot{\alpha} \wedge \ddot{\alpha}|}{|\dot{\alpha}||\dot{\alpha}\wedge \ddot{\alpha } \wedge \dddot{\alpha}|^{2}} \\
\end{gather}\]

\[\begin{gather}
\dot{\alpha}(t) = (1, t-1, t^{2}, t^{3})^{T} \\
\ddot{\alpha}(t) = (0, 1, 2t, 3t^{2})^{T} \\
\dddot{\alpha}(t) = (0, 0, 2, 6t)^{T} \\
\alpha^{(4)}(t) = (0, 0, 0, 6)^{T} \\
\end{gather}\]

\[\begin{gather}
\dot{\alpha}(0) = (1, -1, 0, 0)^{T} \\
\ddot{\alpha}(0) = (0, 1, 0, 0)^{T} \\
\dddot{\alpha}(0) = (0, 0, 2, 0)^{T} \\
\alpha^{(4)}(0) = (0, 0, 0, 6)^{T} \\
\end{gather}\]

\[\begin{gather}
|\dot{\alpha}|^{2} = 2 \implies |\dot{\alpha}|^{3} = 2\sqrt{ 2 } \\
|\dot{\alpha} \wedge\ddot{\alpha}|^{2} = \begin{vmatrix}
2 & -1 \\
-1 & 1
\end{vmatrix} = 1 \implies |\dot{\alpha} \wedge \dot{\alpha}| = 1 \\
|\dot{\alpha} \wedge \ddot{\alpha} \wedge \dddot{\alpha}|^{2} = \begin{vmatrix}
2 & -1 & 0 \\
-1 & 1 & 0 \\
0 & 0 & 4
\end{vmatrix} = 4 \implies |\dot{\alpha} \wedge \ddot{\alpha} \wedge \dddot{\alpha}| = 2 \\
*(\dot{\alpha} \wedge \ddot{\alpha} \wedge \dddot{\alpha} \alpha^{(4)}) = \begin{vmatrix}
1 & 0 & 0 & 0 \\
-1 & 1 & 0 & 0 \\
0 & 0 & 2 & 0 \\
0 & 0 & 0 & 6
\end{vmatrix} = 12
\end{gather}\]

\[\begin{gather}
k_{1} = \frac{1}{2\sqrt{ 2 }} \\
k_{2} = \frac{2}{1} = 2 \\
k_{3} = \frac{12\cdot 1}{\sqrt{ 2 } \cdot 4} = \frac{3}{\sqrt{ 2 }}
\end{gather}\]

\subsubsection{Задача на
``галочки''}\label{ux437ux430ux434ux430ux447ux430-ux43dux430-ux433ux430ux43bux43eux447ux43aux438}

Выписать инварианты магнитного поля \(F_{\mu \nu}F^{\mu \nu}\) и
\(\frac{1}{2}\epsilon^{\mu \nu \rho \sigma}F_{\mu \nu}F_{\sigma \rho}\)
в терминах напряжений электронных и магнитных полей \(E\) и \(H\).

\[g^{\mu \nu} = g_{\mu \nu} = \begin{pmatrix}
1 & 0 & 0 & 0 \\
0 & -1 & 0 & 0 \\
0 & 0 & -1 & 0 \\
0 & 0 & 0 & -1
\end{pmatrix}\]

\[F_{\mu \nu} = \begin{pmatrix}
0 & -E_{1} & -E_{2} & -E_{3} \\
E_{1} & 0 & H_{3} & -H_{2} \\
E_{2} & -H_{3} & 0 & H_{1} \\
E_{3} & H_{2} & -H_{1} & 0
\end{pmatrix}\]

\[\begin{gather}
F_{\mu \nu}F^{\mu \nu} = g_{\mu \mu}F^{\mu}_{\nu}F^{\mu \nu} = g_{\mu \mu}g_{\nu \nu}(F^{\mu \nu})^{2} = \\
= 2(-E_{1}^{2} -E_{2}^{2} - E_{3}^{2} + H_{3}^{2} + H_{2}^{2} + H_{1}^{2})
\end{gather}\]

\[\begin{gather}
\frac{1}{2}\epsilon^{\mu \nu \rho\sigma}F_{\mu \nu}F_{\sigma \rho} = -4(E_{1}H_{1} +E_{2}H_{2} + E_{3}H_{3}) \\
\end{gather}\]
\end{document}
