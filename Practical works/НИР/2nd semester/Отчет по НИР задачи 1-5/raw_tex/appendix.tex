\newpage
\phantomsection
\section{Приложение}

\RaggedLeft

\phantomsection
\subsection*{Приложение 1. Код решения задачи 1}
\label{appendix:1}
\codefromfile{work_1.R}{R}

\newpage
\phantomsection
\subsection*{Приложение 2. Код решения задачи 2}
\label{appendix:2}
\codefromfile{work_2.R}{R}

\newpage
\phantomsection
\subsection*{Приложение 3. Код решения задачи 3}
\label{appendix:3}
\codefromfile{work_3.R}{R}

\newpage
\phantomsection
\subsection*{Приложение 4. Код решения задачи 4}
\label{appendix:4}
\codefromfile{work_4.py}{python}

\newpage
\phantomsection
\subsection*{Приложение 5. Код решения задачи 5}
\label{appendix:5}
\codefromfile{work_5.py}{python}


\newpage
\phantomsection
Приложение 6
\image{images/plot_2.png}{Схема усеченного решающего дерева, полученного при решении задачи 4.}{1}
\label{appendix:6}

\newpage
\phantomsection
Приложение 7
\begin{code}
\begin{minted}{python}
numerical_features = ['first_occurrence_date', 'last_occurrence_date',
                      'reported_date', 'geo_x', 'geo_y', 'geo_lat', 'geo_lon',
                      'victim_count', 'reported_time', 'reported_time_utc']
numerical_data = pd.DataFrame(index=data.index, columns=numerical_features)
numerical_data.loc[:, numerical_features] = data.loc[:,
    data.columns.isin(numerical_features)]

time_features = ['first_occurrence_date',
                 'last_occurrence_date', 'reported_date']

for col in time_features:
    numerical_data.loc[:, col] = numerical_data[col].apply(
        lambda timestr: pd.to_datetime(timestr, format='%m/%d/%Y %I:%M:%S %p'))
numerical_data['reported_time'] = \
    numerical_data['reported_date'].apply(lambda x: pd.to_datetime(
        x.strftime('%H:%M:%S')))
numerical_data.loc[:, 'reported_time_utc'] = \
    numerical_data['reported_time'].dt.hour * 60 * 60 + \
    numerical_data['reported_time'].dt.minute * 60 + \
    numerical_data['reported_time'].dt.second
numerical_data = numerical_data.astype('float64', errors='ignore')
# Отбросим значения даты и времени в формате Timestamp для нормализации
normalizible_features = ['geo_x', 'geo_y', 'geo_lat',
                         'geo_lon', 'victim_count', 'reported_time_utc']
normalizible_data = numerical_data.loc[:, numerical_data.columns.isin(
    normalizible_features)]
\end{minted}
\captionof{listing}{Нормализация числовых данных, представленных в датасете.}
\end{code}
\label{appendix:7}

\newpage
\phantomsection
Приложение 8
\begin{codeappendix}
\begin{minted}{python}
geo_x                1.000000
geo_y                0.988746
geo_lat             -0.989791
geo_lon              0.999873
victim_count         0.000309
reported_time_utc   -0.004854
Name: geo_x, dtype: float64 
geo_x                0.988746
geo_y                1.000000
geo_lat             -0.962731
geo_lon              0.988911
victim_count         0.001351
reported_time_utc   -0.003204
Name: geo_y, dtype: float64 
geo_x               -0.989791
geo_y               -0.962731
geo_lat              1.000000
geo_lon             -0.987968
victim_count         0.000482
reported_time_utc    0.005366
Name: geo_lat, dtype: float64 
geo_x                0.999873
geo_y                0.988911
geo_lat             -0.987968
geo_lon              1.000000
victim_count         0.000331
reported_time_utc   -0.004929
Name: geo_lon, dtype: float64 
geo_x                0.000309
geo_y                0.001351
geo_lat              0.000482
geo_lon              0.000331
victim_count         1.000000
reported_time_utc   -0.003938
Name: victim_count, dtype: float64 
geo_x               -0.004854
geo_y               -0.003204
geo_lat              0.005366
geo_lon             -0.004929
victim_count        -0.003938
reported_time_utc    1.000000
Name: reported_time_utc, dtype: float64 
\end{minted}
\captionof{listing}{Вывод \hyperref[code:41]{кода 41} (Нахождение коррелирующих признаков, представленных в датасете).}
\end{codeappendix}
\label{appendix:8}

\newpage
\phantomsection
Приложение 9
\begin{codeappendix}
\begin{minted}{python}
clean_features = ['offence_code+extension', 'reported_time_utc',
                  'geo_x', 'geo_y', 'district_id', 'is_crime',
                  'is_traffic', 'victim_count']
clean_data = pd.DataFrame(columns=clean_features, index=data.index)

clean_data['offence_code+extension'] = data['offense_code'] * \
    10 + data['offense_code_extension']
clean_data['reported_time_utc'] = numerical_data['reported_time_utc']
clean_data['geo_x'] = numerical_data['geo_x']
clean_data['geo_y'] = numerical_data['geo_y']
clean_data['district_id'] = data['district_id']
clean_data['is_crime'] = data['is_crime']
clean_data['is_traffic'] = data['is_traffic']
clean_data['victim_count'] = data['victim_count']
clean_data = clean_data.dropna()
# print(f'Number of objects in cleaned dataset: {len(clean_data.index)}')

normalizible_clean_features = ['reported_time_utc', 'geo_x', 'geo_y',
                               'victim_count']
normalizible_clean_data = clean_data.loc[:, clean_data.columns.isin(
    normalizible_clean_features)]
std_scaler = preprocessing.StandardScaler()
normalized_clean_data = pd.DataFrame(
    std_scaler.fit_transform(normalizible_clean_data),
    columns=normalizible_clean_data.columns,
    index=normalizible_clean_data.index)

for col in normalizible_clean_features:
    clean_data[col] = normalized_clean_data.loc[:, col]
clean_data = clean_data.apply(
    lambda x: pd.to_numeric(x, errors='coerce')).dropna()

clean_data.to_csv('./clean_crime.csv')
\end{minted}
\captionof{listing}{Подготовка данных к применению метода PCA.}
\end{codeappendix}
\label{appendix:9}

\newpage
\phantomsection
Приложение 10
\image{images/plot_5.png}{График доли дисперсии объясненной каждой из компонент после применения метода PCA.}{1}
\label{appendix:10}

\newpage
\phantomsection
Приложение 11
\image{images/plot_8.png}{Двумерная визуализация данных, полученная после применения алгоритма t-SNE (окраска по критерию \textit{victim\_count}).}{1}
\label{appendix:11}

\newpage
\phantomsection
Приложение 12
\image{images/plot_9.png}{Двумерная визуализация данных, полученная после применения алгоритма t-SNE (окраска по критерию \textit{offence\_code+extension}).}{1}
\label{appendix:12}
